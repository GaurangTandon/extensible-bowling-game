\section{New features}

\subsection{Maximum player}

This was simple to implement. We only changed the number of maximum members in Alley.

\subsection{Ad-hoc queries on user data}

\begin{itemize}
\item We have provided a queries button in the UI of the main panel, using which user can invoke a panel which allows for three types of queries: (1) Best scorer (2) Worst scorer (3) Highest cumulative score so far
\item This was made possible by using the Utilities.Score history DAT file, which already logs all the previous scores.
\item New classes implemented:
    \begin{itemize}
        \item ControlDesk.AdhocView.java: for managing the View of the queries
        \item Utilities.ScoreHistoryFile.java: for managing the backend of the actual queries.
    \end{itemize}
\end{itemize}

\subsection{Pause and resume games on a Lane.Lane}

\begin{itemize}
\item This was relatively simple to implement as we have split up Lane.Lane into Lane.Lane, Bowlers.ScorableParty, and Bowlers.ScorableBowler. Each of these classes now have a \code{saveState} and \code{loadState} method.
\item Every time the lane is paused, first the Bowlers.ScorableParty calls saveState on each Bowlers.ScorableBowler it has. Each bowler then saves its own state in order. Finally, the party then saves certain specific information. Load state proceeds in a similar fashion.
\item As you can see, all the three \code{saveState} calls are \textbf{decoupled} from each other. We can happily change the logic in one class, and it would not affect the other classes in any way.
\item As for the frontend, we have provided two extra buttons in Lane.LaneStatusView.java that lets the users pause and resume the game.
\end{itemize}
