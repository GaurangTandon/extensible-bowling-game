\section{Introduction}

\subsection{Short overview}

\paragraph{} Hey there, welcome to our Bowling Game project! We were hired specifically \textit{not} to build a new system, but to \textbf{fix} an original implementation by some other team in terms of its \textbf{maintainability}, and implement only a few new features into it. We have toiled hard to deliver a codebase which satisfies many important software design principles.
\paragraph{} We have also included all the three features, namely:
\begin{itemize}
    \item Maximum multiplayer up to six players
    \item Pause and play the game, and resume a paused game from memory
    \item View statistical information about the best and worst players
\end{itemize}

It was interesting to note that our refactoring was effective enough that implementing all these features required small changes to the code at obvious locations.

\subsection{How we stuck to software engineering principles}

Throughout our refactoring, we have consistently upholded major principles of software design. The first is obviously \textbf{DRY} - Don't Repeat Yourself. This is a common principle, wherein we make sure that the same functionality is not duplicated in more than one place. In our project, we have ensured that even the simplest of logics, like checking whether a strike occurred or not (\code{pinsDown == 10}) has been carefully placed in a dedicated method so as to not duplicate it across mutliple places.

Another principle we stuck to was \textbf{Single Responsibility principle}: which states that every class should be responsible for exactly one purpose. Note that this was the hardest to sustain, since the original codebase had a God class (Lane.java) by itself.

\textbf{Law of Demeter} was another arena we conquered. We ensured that our classes are kept independent of each other, that connections between different classes are minimized (\textbf{coupling}), and that related classes are kept together (\textbf{cohesion}).

We have also extensively used \textbf{inheritance} wherever required to convey the semantic logic of our classes. For example, \code{BowlerInfo} acts as a base class for \code{Bowler}. The sole responsibility of BowlerInfo is to keep track of metadata of a Bowler, whereas the sole role of a Bowler is to be able to roll balls down a lane. We then implemented \code{ScorableBowler} which inherits from Bowler, and also adds the ability to keep track of scores across games. This chain of inheritance ensures that our classes have cohesive usage of attributes as well as reusability throughout.

\end{itemize}

\subsection{Contributions}

Each of us worked for almost 24-30 hours on the codebase. We had these roles:

\begin{itemize}
    \item Gaurang: Complete Refactor (Specially reducing cyclomatic complexity, implementing observer pattern)
    \item Animesh: UI based refactors (from the View Classes), miscellaneous fixes
    \item Avani: Building Documentation and fixing several miscellaneous code quality errors
\end{itemize}
